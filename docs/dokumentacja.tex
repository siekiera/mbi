\documentclass[a4paper,12pt]{article}
\usepackage[polish]{babel}
\usepackage[utf8]{inputenc}
\usepackage[T1]{fontenc}
\usepackage[intlimits]{amsmath}
%\usepackage{fullpage}
\title{Metody bioinformatyki - dokumentacja wstępna projektu}
\author{Michał Toporowski \and Marcin Swend}
\date{\today}

\newcommand{\plw}[1]{\frac{#1}{2}}

\begin{document}
\maketitle
%\setlength{\parindent}{0pt}
\section{Cel projektu}
Stworzenie interaktywnej aplikacji pozwalającej na zrozumienie działania algorytmu obliczającego macierz podobieństwa metodą BLOSUM.

\section{Założenia}
\begin{itemize}
\item aplikacja osadzona w przeglądarce 
\item wykorzystanie języka JavaScript i HTML5
\item \textbf{dane wejściowe:} sekwencje wprowadzane przez użytkownika z klawiatury lub wczytywane z pliku
\item \textbf{dane wyjściowe:} macierz podobieństwa dla danych sekwencji obliczona metodą BLOSUM oraz pośrednie wyniki obliczeń wyświetlane użytkownikowi poprzez przeglądarkę
\item możliwość śledzenia przez użytkownika kolejnych kroków algorytmu obliczającego macierz
\begin{itemize}
\item opis aktualnie wykonywanego kroku wraz z uzasadnieniem wyniku
\item wyróżnienie aktualnie obliczanej komórki/kolumny
\item możliwość przechodzenia do wybranych kroków i etapów algorytmu (bezpośrednie przejście do kolejnego etapu, powrót do poprzedniego kroku/etapu)
\end{itemize}
\end{itemize}
\section{Algorytm}
Wykorzystany zostanie algorytm składający się z następujących etapów:
\begin{enumerate}
\item \textbf{Obliczenie macierzy podstawień}\\
Obliczenie macierzy $A_{i,j}$ zawierającej częstości występowania symbolu $i$ w danym miejscu na łańcuchu oraz symbolu $j$ w tym samym miejscu w innych łańcuchach.
\item \textbf{Estymacja prawdopodobieństwa wystąpienia pary}\\
Obliczenie macierzy zawierającej prawdopodobieństwa wystąpienia pary $q_{i,j}$
na podstawie wzoru:
$$q_{i,j} = \frac{A_{i,j}}{\sum_{x, y}A_{x,y}}$$
\item \textbf{Estymacja prawdopodobieństw symboli}\\
Obliczenie prawdopodobieństw wystąpienia każdego z symboli $p_i$ na podstawie wzoru:
$$p_i = q_{i,i} + \sum_{j \neq i}\frac{q_{i,j}}{2}$$
\item \textbf{Obliczenie macierzy BLOSUM}\\
Obliczenie macierzy $e_{i,j}$ na podstawie wzoru:
$$e_{i,j} = 2\log_2\left(\frac{q_{i,j}}{p_i p_j}\right)$$
Następnie obliczenie macierzy BLOSUM poprzez zaokrąglenie do liczb całkowitych.
\end{enumerate}
\end{document}